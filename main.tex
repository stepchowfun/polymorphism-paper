\documentclass[acmlarge]{acmart}

\usepackage{booktabs} % For formal tables


\usepackage[ruled]{algorithm2e} % For algorithms
\renewcommand{\algorithmcfname}{ALGORITHM}
\SetAlFnt{\small}
\SetAlCapFnt{\small}
\SetAlCapNameFnt{\small}
\SetAlCapHSkip{0pt}
\IncMargin{-\parindent}

% Metadata Information
\acmJournal{JOCCH}
\acmVolume{9}
\acmNumber{4}
\acmArticle{39}
\acmYear{2010}
\acmMonth{3}
\acmArticleSeq{11}

%\acmBadgeR[http://ctuning.org/ae/ppopp2016.html]{ae-logo}
%\acmBadgeL[http://ctuning.org/ae/ppopp2016.html]{ae-logo}


% Copyright
%\setcopyright{acmcopyright}
%\setcopyright{acmlicensed}
%\setcopyright{rightsretained}
%\setcopyright{usgov}
\setcopyright{usgovmixed}
%\setcopyright{cagov}
%\setcopyright{cagovmixed}

% DOI
\acmDOI{0000001.0000001}

% Paper history
\received{February 2007}
\received{March 2009}
\received[accepted]{June 2009}


% Document starts
\begin{document}
% Title portion
\title{A Multifrequency MAC Specially Designed for Wireless Sensor
  Network Applications}
\author{Gang Zhou}
\orcid{1234-5678-9012-3456}
\affiliation{%
  \institution{College of William and Mary}
  \streetaddress{104 Jamestown Rd}
  \city{Williamsburg}
  \state{VA}
  \postcode{23185}
  \country{USA}}
\author{Yafeng Wu}
\affiliation{%
  \institution{University of Virginia}
  \department{School of Engineering}
  \city{Charlottesville}
  \state{VA}
  \postcode{22903}
  \country{USA}
}
\author{Ting Yan}
\affiliation{%
  \institution{Eaton Innovation Center}
  \city{Prague}
  \country{Czech Republic}}
\author{Tian He}
\affiliation{%
  \institution{University of Minnesota}
  \country{USA}}
\author{Chengdu Huang}
\author{John A. Stankovic}
\author{Tarek F. Abdelzaher}
\affiliation{%
  \institution{University of Virginia}
  \department{School of Engineering}
  \city{Charlottesville}
  \state{VA}
  \postcode{22903}
  \country{USA}
}


\begin{abstract}
Multifrequency media access control has been well understood in
general wireless ad hoc networks, while in wireless sensor networks,
researchers still focus on single frequency solutions. In wireless
sensor networks, each device is typically equipped with a single
radio transceiver and applications adopt much smaller packet sizes
compared to those in general wireless ad hoc networks. Hence, the
multifrequency MAC protocols proposed for general wireless ad hoc
networks are not suitable for wireless sensor network applications,
which we further demonstrate through our simulation experiments. In
this article, we propose MMSN, which takes advantage of
multifrequency availability while, at the same time, takes into
consideration the restrictions of wireless sensor networks. Through
extensive experiments, MMSN exhibits the prominent ability to utilize
parallel transmissions among neighboring nodes. When multiple physical
frequencies are available, it also achieves increased energy
efficiency, demonstrating the ability to work against radio
interference and the tolerance to a wide range of measured time
synchronization errors.
\end{abstract}


%
% The code below should be generated by the tool at
% http://dl.acm.org/ccs.cfm
% Please copy and paste the code instead of the example below.
%
\begin{CCSXML}
<ccs2012>
 <concept>
  <concept_id>10010520.10010553.10010562</concept_id>
  <concept_desc>Computer systems organization~Embedded systems</concept_desc>
  <concept_significance>500</concept_significance>
 </concept>
 <concept>
  <concept_id>10010520.10010575.10010755</concept_id>
  <concept_desc>Computer systems organization~Redundancy</concept_desc>
  <concept_significance>300</concept_significance>
 </concept>
 <concept>
  <concept_id>10010520.10010553.10010554</concept_id>
  <concept_desc>Computer systems organization~Robotics</concept_desc>
  <concept_significance>100</concept_significance>
 </concept>
 <concept>
  <concept_id>10003033.10003083.10003095</concept_id>
  <concept_desc>Networks~Network reliability</concept_desc>
  <concept_significance>100</concept_significance>
 </concept>
</ccs2012>
\end{CCSXML}

\ccsdesc[500]{Computer systems organization~Embedded systems}
\ccsdesc[300]{Computer systems organization~Redundancy}
\ccsdesc{Computer systems organization~Robotics}
\ccsdesc[100]{Networks~Network reliability}

%
% End generated code
%

% We no longer use \terms command
\terms{Design, Algorithms, Performance}

\keywords{Wireless sensor networks, media access control,
multi-channel, radio interference, time synchronization}


\thanks{This work is supported by the National Science Foundation,
  under grant CNS-0435060, grant CCR-0325197 and grant EN-CS-0329609.

  Author's addresses: G. Zhou, Computer Science Department, College of
  William and Mary; Y. Wu {and} J. A. Stankovic, Computer Science
  Department, University of Virginia; T. Yan, Eaton Innovation Center;
  T. He, Computer Science Department, University of Minnesota; C.
  Huang, Google; T. F. Abdelzaher, (Current address) NASA Ames
  Research Center, Moffett Field, California 94035.}


\maketitle

% The default list of authors is too long for headers}
\renewcommand{\shortauthors}{G. Zhou et al.}



\section{Introduction}

Hello, World!

\section{Conclusion}

Hello, World!

\section{Supplementary materials}

\begin{acks}

The authors would like to thank nobody!

\end{acks}

% Bibliography
\bibliographystyle{ACM-Reference-Format}
\bibliography{bibliography}



\end{document}
